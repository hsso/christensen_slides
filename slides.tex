\documentclass{beamer}

\usepackage[version=3]{mhchem}
\usepackage{siunitx,booktabs,multirow,graphicx,natbib}
\usepackage{aas_macros,hyperref}

\graphicspath{{/home/miguel/HssO/Christensen/paper/}}

\DeclareSIUnit\Kms{\milli\kelvin\km\per\s}
\DeclareSIUnit\kms{\km\per\s}
\DeclareSIUnit\kgs{\kg\per\s}
\DeclareSIUnit\mols{{molec.}\,s^{-1}}
\sisetup{
  separate-uncertainty,
}
\newcommand{\herschel}{{\it Herschel}}
\newcommand{\christensen}{C/2006~W3 (Christensen)}
\newcommand{\rh}{r_\mathrm{h}}
\newcommand{\trans}{$J_{K_\mathrm{a} K_\mathrm{c}}\ (1_{10}\text{--}1_{01})$}
\newcommand{\nh}{$J_K\ (1_{0}\text{--}0_{0})$}

\newcommand\Fontvi{\fontsize{6}{7.2}\selectfont}

\newcommand{\wbsho}{11.4(35)}
\newcommand{\hrsho}{16.2(46)}
\newcommand{\wbsv}{-71(40)}
\newcommand{\wbshv}{-240(90)}
\newcommand{\wbsvv}{-10(70)}
\newcommand{\wbsdv}{-230(120)}
\newcommand{\hrsv}{-77(67)}
\newcommand{\hrshv}{-230(120)}
\newcommand{\hrsvv}{-150(66)}
\newcommand{\hrsdv}{-80(140)}
\newcommand{\wbsqho}{1.7(5)e27}
\newcommand{\hrsqho}{2.3(7)e27}
\newcommand{\qho}{2.0(5)e27}

\newcommand{\wbsn}{<13}
\newcommand{\hrsn}{<17}
\newcommand{\wbsqn}{<1.5e27}
\newcommand{\hrsqn}{<1.9e27}
\newcommand{\qnh}{<1.5e27}

\begin{document}

\title{\herschel{} observations of gas and dust in comet
\christensen{} at  5 AU from the Sun}
\date{HssO Paris meeting, 2014}

\frame{\titlepage}

\begin{frame}
\frametitle{\christensen{} observations}
  \begin{itemize}
  \item Searched for emission in the \ce{H2O} and \ce{NH3}
  ground-state rotational transitions, \trans{} at
  \SI{557}{\giga\hertz} and \nh{} at \SI{572}{\giga\hertz}
  simultaneously with HIFI
  \item Photometric observations of the dust coma in the
  \SIlist{70;160}{\um} channels were acquired with PACS
  \end{itemize}
\end{frame}

\begin{frame}
\frametitle{\herschel{} \christensen{} observations}
  \begin{itemize}
  \item long-period comet that was discovered in November 2006 at a distance of
8.6 AU from the Sun.
  \item It passed perihelion on 6 July 2009 at a heliocentric distance of 3.13
AU.
  \end{itemize}
  \Fontvi
  \begin{table}
    \centering
    \begin{tabular}{c S[table-format = 3] c
		    S[table-format = 10,
		    table-space-text-post = \textsuperscript{\emph{h}}]
		    S[table-format = 2.1]
		    S[table-format = 3]
		    c c c c S
		    }
      \toprule
      Date &
      {OD} &
      Inst. & {ObsID} & {Exp.} &
      {Angle} &
      {Scan size} &
      {Speed} &
      $\rh$ &
      $\Delta$ &
      {$\phi$}\\
      (yyyy-mm-dd.ddd) & & & & {(\si{\minute})} &
      {(\si{\degree})} & {($\si{\arcmin} \times \si{\arcmin}$)} &
      {("/s)} &
      (AU) & (AU) & {(\si{\degree})}\\
      \midrule
      \input{log_slide.txt}
      \bottomrule
    \end{tabular}
  \end{table}
\end{frame}

\begin{frame}
\frametitle{HIFI observations}
\includegraphics[width=.5\textwidth]{figures/christensen-figure0.pdf}
\includegraphics[width=.5\textwidth]{figures/christensen-figure1.pdf}

A tentative detection of the ortho-\ce{H2O} ground-state transition is
observed in the HIFI spectra with 4-$\sigma$ significance
\end{frame}

\begin{frame}
\frametitle{Production rates}
\begin{table}
  \label{tbl:q}
  \centering
  \begin{tabular}{ccc
		  S[table-format = <2.1(2)]
		  S[table-format = +2(2)]
		  S[table-format = <1.1(1)e2]
		  }
    \toprule
    Molecule & Spec. &
    $\sigma_{T_\mathrm{mB}}$ &
    ${\int T_\mathrm{mB}\, dv}$ &
    {$\Delta v$} &
    {$Q$}\\
    & & (\si{\milli\K}) & {(\si{\Kms})} & {(\si{\m\per\s})} &
    {(\si{\mols})} \\
    \midrule
    \multirow{2}{*}{\ce{H2O}} & WBS & 1.5 & \wbsho & \wbsv & \wbsqho \\
                              & HRS & 6.3 & \hrsho & \hrsv & \hrsqho \\
    \midrule
    \multirow{2}{*}{\ce{NH3}} & WBS & 1.5 & \wbsn & & \wbsqn \\
                              & HRS & 6.3 & \hrsn & & \hrsqn \\
    \bottomrule
  \end{tabular}
\end{table}
\end{frame}

\begin{frame}
\frametitle{Outgassing evolution}
\includegraphics[width=.8\textwidth]{figures/christensen-figure5.pdf}

Water production rate measured by HIFI and the production rate at
the subsolar point on the nucleus shown by the solid curve agree
\end{frame}

\begin{frame}
\frametitle{\ce{H2O} and \ce{NH3} production rates}
\begin{itemize}
\item The \ce{H2O} line is only marginally detected
\item derived \ce{H2O} production rate and line shape and velocity shift are
consistent with the emission feature having a cometary origin
\item We derive a water production rate of \SI{\qho}{\mols} using a spherically
symmetric radiative transfer model
\item we aimed to detect the ground-state rotational transitions of
ortho-\ce{NH3} in the USB of HIFI's band 1b
\item a 3-$\sigma$ upper limit for the ammonia production rate of
\SI{\qnh}{\mols} is derived, corresponding to a mixing ratio
$Q_\ce{NH3}/Q_\ce{H2O} < 0.75$
\end{itemize}
\end{frame}

\begin{frame}
\frametitle{Dust emission}
\includegraphics[width=.8\textwidth]{figures/christensen-figure6.pdf}
\end{frame}

\begin{frame}
\frametitle{Dust production rates}
\begin{itemize}
\item To determine the dust production rate $Q_\mathrm{dust}$, we
compared the flux densities measured on the brightest pixels of the blue
and red PACS maps with those expected from a model of dust thermal
emission
\item The model used for this study is the same as that applied to the PACS
data obtained in 2009 at 3.35 AU from the Sun
\citep{2010A&A...518L.149B}
\item Absorption cross-sections calculated with the
Mie theory were used to compute the temperature of the grains solving the
equation of radiative equilibrium
\end{itemize}
\end{frame}

\begin{frame}
\frametitle{Radial profiles}
\includegraphics[width=.8\textwidth]{figures/christensen-figure7.jpg}

We fit the radial dependence of the surface brightness with radially
symmetric profiles for the blue and red bands.
\end{frame}

\begin{frame}
\frametitle{Results}
\begin{itemize}
  \item The blueshift of the water line detected by HIFI suggests
  preferential emission from the subsolar point.
  \item Also
  possible that water sublimation occurs in small ice-bearing grains
  that are emitted from an active region on the nucleus surface at a
  speed of \SI{\sim 0.2}{\kms}.
  \item Dust production rates derived in
  August 2010 are roughly one order of magnitude lower than in
  September 2009
  \item Dust-to-gas production rate
  ratio remained approximately constant during the period when the
  activity became increasingly dominated by CO outgassing.
\end{itemize}
\end{frame}

\begin{frame}
\frametitle{References}
\bibliographystyle{aa}
\bibliography{ads}
\end{frame}

\end{document}
